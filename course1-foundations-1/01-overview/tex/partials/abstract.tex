Quantitative social science research increasingly demands sophisticated computational capabilities. However, existing research 
practices remain constrained by a structural misalignment between the epistemic organization of the research lifecycle and 
available computational tools, resulting in fragmented tool and data ecosystems, 
limited workflow integration, steep learning curves, 
growing data complexity, and persistent reproducibility challenges.

We define five problems in modern quantitative social science research.
 This paper addresses the following research question: How can the epistemic structure of quantitative social science research be 
systematically encoded into an integrative, modular, and reproducible computational framework for tackling the defined problems in modern quantitative social science research?
 To answer this question, 
we present a comprehensive software research framework built around a unified, modular architecture that supports the 
full research lifecycle, including literature review and meta-analysis, experimental and survey research, document analysis, and 
secondary data analysis.

The core idea of the proposed framework is to introduce a shared and integrative structural layer that mediates between heterogeneous data sources, 
analytical methods, and research outputs. 
By defining consistent intermediate representations of research data and analytical states, the framework enables different components 
of the workflow to interoperate without requiring direct coupling between specific tools. 
This approach supports extensible pipelines that can accommodate multiple data modalities, analytical environments,
 and target outputs such as figures, tables, and LaTeX manuscripts, while reducing redundancy, 
 manual transformation, and workflow fragmentation.

We present the system's preliminary design and implementation, demonstrate its use through a representative case study, 
and discuss current limitations and future development directions. The source code is publicly available at:
 \url{https://github.com/wwwwanhonghuang/Quantitative-Social-Science-Research-framework}.

