\subsection{Functional Modules and Capabilities}

The framework provides integrated support for research activities through eight functional modules organized into four categories: core research modules addressing fundamental 
data operations, research support modules providing specialized tools, intelligent assistance augmenting researcher capabilities, and cross-cutting infrastructure enabling system-wide 
properties.

\subsubsection{Core Research Modules}

Core research modules are designed to address fundamental operations that span multiple lifecycle stages and research schemes.

The \textit{Data Collection \& Review} module is designed to provide systematic access to scholarly literature and research data across multiple sources. 
For literature review and meta-analysis, the module aims to offer unified interfaces to academic databases including arXiv~\footnote{https://arxiv.org}, 
PubMed~\footnote{https://pubmed.ncbi.nlm.nih.gov}, CrossRef~\footnote{https://www.crossref.org}, OSF~\footnote{https://osf.io}, SocArXiv~\footnote{https://osf.io/preprints/socarxiv}, 
and Zenodo~\footnote{https://zenodo.org}. The design includes automated metadata extraction, bibliography management with deduplication, and structured storage conventions intended to 
enable systematic reuse of collected materials. For secondary data analysis, the module is designed to provide explicit conventions for data ingestion from external sources and storage 
in unified formats suitable for subsequent processing and analysis. Current implementation covers basic literature retrieval from major databases with metadata extraction, 
while advanced features such as intelligent deduplication and cross-database query optimization remain under development.

The \textit{Data Processing} module is designed to implement configurable transformation pipelines that handle data cleaning, validation, normalization, feature engineering, and 
integration across sources. Processing pipelines are intended to be specified through contracts that define transformation sequences, validate intermediate outputs against schema 
constraints, and maintain comprehensive provenance tracking linking processed data to raw inputs and transformation parameters. The modular processor architecture is designed to 
allow researchers to compose custom pipelines from standard operations including missing data handling, outlier detection and treatment, variable recoding and transformation, 
normalization and standardization, and feature construction. The design emphasizes preservation of audit trails documenting decisions and parameter choices, supporting sensitivity 
analysis and methodological transparency. Current implementation provides limited transformation operations, with features such as advanced validation logic and automated 
quality assessment planned for future development.

The \textit{Data Analysis} module is designed to provide comprehensive statistical and computational methods spanning the diverse analytical approaches documented in 
Section~\ref{sec:data-analysis-methods}. The module aims to organize methods into conceptually coherent categories while exposing them through consistent interfaces that abstract implementation details. 
The design encompasses statistical inference methods including descriptive statistics, hypothesis testing, and various regression models; causal inference methods including propensity score 
matching, instrumental variables, difference-in-differences, regression discontinuity designs, and synthetic control methods; machine learning methods for supervised and unsupervised 
learning; and network analysis capabilities. All analytical procedures are designed to accept standardized input formats defined by contracts, produce structured output including effect estimates 
and uncertainty quantification, and maintain execution metadata enabling result traceability. 
Current implementation covers fundamental statistical methods including regression models and basic causal inference techniques, with machine learning and network analysis 
capabilities under active development.

The \textit{Data Visualization} module is designed to generate publication-ready figures and exploratory graphics supporting both analytical investigation and result presentation. 
The design encompasses diverse visualization types including statistical plots, network visualizations, temporal visualizations, geospatial visualizations, and specialized visualizations 
for text analysis. All visualizations are intended to integrate automatically with LaTeX for manuscript inclusion, follow customizable style templates aligned with publication standards, 
and support both static high-resolution outputs and interactive exploration. Visualization specifications are designed to be definable in contracts, enabling automated figure regeneration 
when underlying data or analyses change. 
Current implementation provides core statistical plots and basic LaTeX integration, with interactive visualization capabilities and advanced customization options planned 
for future releases.


\subsubsection{Research Support Modules}

Research support modules provide specialized capabilities for particular research activities.

The \textit{Questionnaire Design} module supports primary data collection through structured survey instruments. 
The module enables complex questionnaire definition with skip logic, validation rules, and response format specifications. Questionnaires are rendered using LaTeX for 
professional presentation in print or PDF formats, ensuring typographic quality appropriate for academic research. The module maintains consistency between questionnaire definitions 
and resulting data schemas, automatically generating data dictionaries and codebooks that document variable definitions, response options, and survey structure. 
This integration ensures that collected data conform to expected formats and simplifies subsequent analysis by providing complete metadata.

The \textit{LaTeX Code Generation} module automates production of publication-ready tables and figures from analytical results. The module translates structured analysis 
outputs into LaTeX table environments following disciplinary formatting conventions such as APA or ASA styles, generates figure captions and references, and produces complete LaTeX 
source files that integrate seamlessly with manuscript documents. Automated generation eliminates manual transcription of numerical results, reduces errors, ensures consistency 
between computational outputs and manuscript presentation, and accelerates the transition from analysis completion to manuscript preparation. Templates allow customization 
of formatting details while maintaining structural consistency.

\subsubsection{Intelligent Assistance}

The \textit{LLM Q\&A} module integrates large language models to augment researcher capabilities across multiple stages of the research lifecycle. During problem discovery, 
the module facilitates interactive exploration of research questions, literature synthesis, and hypothesis generation through conversational interfaces. During data processing 
and analysis, the module provides intelligent assistance for tasks such as data quality assessment, method selection guidance, and result interpretation. During dissemination, 
the module assists with manuscript drafting, method description generation, and consistency checking. The module's integration with other framework components enables 
context-aware assistance that leverages information from contracts, data schemas, and workflow specifications to provide relevant suggestions. Importantly, the module 
functions as an augmentative tool that supports researcher judgment rather than replacing human decision-making, with all AI-generated suggestions requiring explicit 
researcher validation before incorporation into research workflows.

\subsubsection{Cross-Cutting Infrastructure}

Cross-cutting infrastructure provides system-wide capabilities that span all functional modules and lifecycle stages.

The \textit{Contract Definition} module serves as the foundational infrastructure supporting contract-driven design throughout the framework. This module provides mechanisms for 
defining contracts in human-readable formats such as YAML or JSON, parsing and validating contract syntax, compiling contracts into executable validation rules, and enforcing contract 
compliance during workflow execution. Contracts defined through this module are used to specify data schemas by defining data structures, data types, and constraint conditions. 
They are also used to describe workflow specifications by defining operation sequences, dependency relationships, and associated parameters. In addition, contracts document analysis 
configurations by recording method selection and parameter settings, define visualization templates by specifying figure types, stylistic conventions, and annotation information, and 
prescribe output formats by determining how results are stored and presented. The module's validation engine continuously checks whether data and 
operations satisfy contract requirements, enabling early detection of constraint violations and preventing the propagation of invalid states throughout the workflow.

Provenance tracking infrastructure maintains comprehensive records of computational operations, data transformations, and analytical decisions throughout the research process. 
For every operation, the system records code versions and commit identifiers, parameter configurations and random seeds, execution timestamps and computational environment 
specifications including software versions and system configurations, and data lineage graphs showing how outputs derive from inputs through transformation sequences. 
This provenance information serves multiple purposes: enabling exact reproduction of analytical results by documenting all factors affecting computation, supporting 
sensitivity analysis by facilitating systematic parameter variation, providing audit trails for methodological transparency and peer review, and enabling debugging by 
tracing errors to their sources in data or code. Provenance metadata is stored in structured formats alongside research outputs, ensuring that reproducibility information 
remains accessible and machine-readable.

Version control integration connects the framework with established version control systems such as Git~\footnote{https://git-scm.com}, enabling workflows to leverage 
standard practices for code management, collaboration, and change tracking. The integration automatically commits workflow specifications, contracts, and analysis scripts 
to version control repositories, tags analytical results with associated code versions, and generates reproducibility metadata linking outputs to exact code states. 
This integration ensures that the complete computational environment supporting research findings remains documented and recoverable.
