

\subsection{Lifecycle-Module Mapping}

The framework design envisions that the six lifecycle stages described in Section~\ref{sec:lifecycle} will be supported through 
coordinated use of multiple functional modules, with particular modules playing primary roles at different stages while remaining accessible throughout the research process. 
This section describes the intended integration patterns, recognizing that full implementation of all coordination mechanisms remains an ongoing process.

During \textit{problem discovery}, the design envisions researchers primarily utilizing the LLM Q\&A module for interactive exploration of research questions and theoretical frameworks, 
complemented by the Data Collection \& Review module for literature discovery and the Data Processing module for preliminary exploration of existing datasets. Contracts may be drafted 
informally at this stage to capture initial conceptualizations of data requirements and analytical approaches, though these specifications typically evolve substantially as understanding 
deepens. Current implementation supports basic literature exploration and preliminary data examination, with AI-assisted ideation capabilities under development.

The \textit{research design} stage is intended to center on formalizing research specifications through the Contract Definition module. The design envisions researchers 
defining data schemas, workflow specifications, validation rules, and output requirements through contracts that serve as executable research protocols. 
The Questionnaire Design module is intended to play a primary role when research involves primary data collection through surveys. 
Current implementation supports basic contract definition for data schemas and simple data processing workflow specifications, 
with advanced validation logic and automated questionnaire generation planned for future development.

During \textit{data collection}, the Data Collection \& Review module supports systematic gathering of literature from academic databases for 
literature review schemes or acquisition of existing datasets for secondary data analysis. The Questionnaire Design module facilitates primary data collection when 
research employs survey methods. All collected data are stored according to schemas specified in contracts, ensuring structural consistency and metadata completeness from the outset. 
Contract-based validation immediately identifies data quality issues or schema violations, enabling early detection and correction of problems that might otherwise propagate through subsequent 
stages.

The \textit{data processing} stage applies transformation pipelines defined through the Data Processing module. Researchers specify sequences of processing operations through contracts 
or interactive configuration, with the system executing transformations while validating outputs against schema constraints and maintaining comprehensive provenance linking 
processed data to raw inputs and transformation parameters. Contract-based validation ensures that processed data meet quality criteria before proceeding to analysis, 
preventing downstream errors from malformed or inconsistent data.

The \textit{analysis} stage leverages the Data Analysis module, which provides access to diverse statistical and computational methods described in 
Section~\ref{sec:data-analysis-methods}. Analyses are configured through contracts specifying methods, parameters, and validation criteria, with the system executing 
computations, storing structured results, 
and maintaining provenance metadata. The Data Visualization module integrates closely with analysis procedures to generate both diagnostic plots supporting 
analytical investigation and substantive visualizations for result presentation. Analysis contracts enable automated reanalysis when underlying data or parameters 
change, supporting sensitivity analysis and robustness checking.

During \textit{dissemination}, the LaTeX Code Generation module automates production of publication-ready tables and figures from analysis results, while the 
Data Visualization module produces final graphics. The LLM Q\&A module assists with manuscript drafting and method description generation. All dissemination materials maintain explicit 
links to underlying analyses, data, and code through provenance metadata, ensuring that published results remain verifiable and reproducible. Contracts governing output formats 
ensure consistency in presentation across multiple result types.

This lifecycle-module mapping illustrates how the framework supports complete research workflows through coordinated use of specialized capabilities while maintaining continuity of 
information flow across stage transitions. The contract-driven architecture ensures that specifications, data, and metadata created in early stages automatically propagate 
to later stages, eliminating manual coordination overhead and reducing opportunities for inconsistencies or errors.
