
\subsection{Architectural Overview}

The framework adopts a three-layer architecture that separates conceptual research structure, functional interfaces, and computational implementations. 
This separation promotes loose coupling, extensibility, and adaptability across diverse research contexts while maintaining coherent support for the complete research lifecycle.

\subsubsection{Three-Layer Architecture}

At the highest level, the \textit{Research Workflow Layer} functions as a conceptual layer representing the progression of research across different lifecycle stages. 
This layer provides an explicit structural representation of research stages, including problem discovery,
research design, data collection, data processing, data analysis, and dissemination.
It also captures the logical relationships among these stages.
The workflow layer serves as an organizing framework that guides how functional capabilities are composed to support particular research schemes.

The \textit{Functional API Layer} exposes high-level functions corresponding to common research activities at each lifecycle stage. These functions provide researcher-facing 
interfaces that abstract underlying computational complexity while offering sufficient configurability for diverse methodological approaches. 
Functions in this layer are defined by their inputs, outputs, and semantic behaviors rather than by specific implementations, enabling alternative 
implementations to satisfy the same functional contract. This abstraction allows the framework to accommodate methodological evolution without disrupting established workflows.

The \textit{Infrastructure Module Layer} contains modular computational components that provide the underlying operational support for functional APIs. 
Modules in this layer implement specific capabilities such as data retrieval from academic databases, statistical analysis procedures, visualization 
generation, or document processing. Modules communicate through well-defined interfaces specifying data formats and operational contracts, enabling independent 
development, testing, and substitution. The modular architecture facilitates extensibility: new modules can be integrated by conforming to interface specifications, and 
existing modules can be replaced with alternative implementations without affecting other system components.

Figure~\ref{fig:system-architecture} illustrates the relationships among these three layers and shows how research lifecycle stages map to functional capabilities supported 
by infrastructure modules.

\begin{figure}[t]
    \centering
    \includegraphics[width=\linewidth]{images/arch.pdf}
    \caption{Three-layer architecture of the quantitative social science research framework. The Research Workflow Layer conceptually represents the six lifecycle stages. The 
    Functional API Layer provides high-level research functions. The Infrastructure Module Layer implements modular computational components. Red arrows illustrate primary mappings between 
    lifecycle stages and functional capabilities, though stages may utilize multiple functions as research needs dictate.}
    \label{fig:system-architecture}
\end{figure}

\subsubsection{Contract-Driven Design Philosophy}

A distinguishing feature of this architecture is its contract-driven design, which formalizes research specifications as explicit, validatable, machine-readable contracts. 
This design philosophy addresses fundamental challenges in research reproducibility and workflow automation.

Contracts in this framework serve multiple interrelated purposes. First, they function as \textit{formal specifications} that document data structure expectations, variable 
definitions, transformation sequences, analysis parameters, and output format requirements. 
In contrast to informal documentation or implicit assumptions embedded in code, contracts provide explicit, checkable statements about what data should look 
like and how operations should behave. 
Second, contracts enable \textit{automated validation and verification} throughout the research process. As data flow through processing pipelines and analytical procedures, 
the framework continuously validates that data structures, value ranges, and semantic constraints specified in contracts are satisfied, detecting errors early and preventing 
invalid operations. Third, contracts support \textit{code generation and automation} by providing machine-readable specifications that drive automated production of boilerplate 
code, data transformation pipelines, and output formatting routines. Fourth, contracts serve as \textit{machine-readable research protocol documentation} that captures methodological 
decisions in forms that both humans and computational systems can interpret, bridging the gap between prose method descriptions and executable workflows.

The notion of contracts draws conceptual inspiration from database integrity constraints, which specify conditions that data must satisfy to maintain consistency and validity. Just as 
database constraints prevent invalid data insertion and ensure referential integrity, research workflow contracts prevent incompatible data operations and ensure that transformations 
preserve required properties. However, contracts in this framework extend beyond data constraints to encompass workflow specifications, analytical parameters, and output requirements, 
providing comprehensive formalization of the research process.

Importantly, contracts are implementation-agnostic regarding their representational format. While the current framework employs YAML and JSON for human-readable contract specification, 
the conceptual role of contracts in defining explicit, validatable specifications is independent of particular syntax or serialization format. Contracts could equally 
be expressed in XML, TOML, custom domain-specific languages, or even structured text files, provided they support formal specification of constraints and enable automated 
validation. The essential characteristic is not the notation but the semantic function: formalizing research specifications as explicit contracts that guide execution and enable verification.

By embedding contracts throughout the research workflow, the framework transforms methodological transparency and reproducibility from aspirational goals requiring separate 
documentation efforts into structural properties enforced by the system architecture. Research specifications become executable documentation that directly drives computational 
operations while remaining human-readable and version-controllable.
