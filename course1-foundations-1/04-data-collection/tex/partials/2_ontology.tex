
%=============================================================================
% 2. ONTOLOGICAL COMMITMENTS IN DATA COLLECTION
%=============================================================================
\section{Ontology of Social Fact and Data}

\subsection{The Definition of Data}

Data is conceptualized differently across disciplines. In statistics, data are 
observations that can be analyzed to reveal patterns~\cite{fisher1925statistical}. 
In computer science, data are discrete, machine-readable representations stored and 
processed algorithmically~\cite{date2003introduction}. In the social sciences, data 
are empirical evidence collected through systematic methods to test 
 hypotheses~\cite{king1994designing}. 
Critical data studies scholars emphasize that data are 
not simply given but actively constructed through processes of selection, 
categorization, and measurement~\cite{gitelman2013raw, bowker1999sorting}.

We adopt a phenomenological perspective: we live in a world of phenomena, the lived 
immediacy of experience. Data constitute an operational representation of these 
phenomena. This datafication process transforms the continuous, qualitative flux of 
experience into discrete, formalizable elements that can be manipulated, analyzed, 
and communicated. As Husserl argued, formal operations require idealization, the 
transformation of intuitive experience into exact, repeatable objects of 
thought~\cite{husserl1970crisis}.

This definition lies in its recognition that phenomena in their 
immediacy are not directly operationable for systematic inquiry. Data provide a 
formal operational space, a symbolic domain where phenomena are re-presented in ways 
that enable knowledge production. This is not merely technical translation but 
ontological transformation: we move from the lifeworld (Lebenswelt) to a constructed 
space of measurable entities. What becomes data, and what resists datafication, is 
never neutral but reflects epistemic choices, power relations, and the limits of 
formalization itself~\cite{vandijck2014datafication}.


\subsection{The Social Facts and Its Deconstruction}
In social science research, it is commonly acknowledged that the main target of 
social science research is \textit{social facts}. In Li's work~\cite{li2019guoji}, 
social facts can be seen as entities and five concepts associated with the entities. 
Specifically, such five concepts are (1) states, (2) processes, (3) properties, 
(4) relations, and (5) events.

For quantitative social science, data serve as operational and formal representations 
of such concepts. 

The transformation from abstract social facts to measurable data 
raises fundamental methodological questions: (1) How do we operationalize abstract concepts 
into concrete variables? (2)  What methods ensure systematic and reliable data collection? 
(3) Which dimensions of social reality resist datafication? (4) Do our measurements validly 
capture the phenomena they claim to represent? 
The question (1) is already answered in research design phase. 
In data collection phase, we mainly focusing on question (2) to (3).


\subsection{Scale and Hierarchy in Spatiotemporal Structures of Social Facts}
Similar to dynamical systems in physics, social facts in quantitative research exhibit 
spatiotemporal structures that are multi-scale and hierarchical. Phenomena unfold across 
different temporal scales (from moments to epochs) and spatial scales (from individuals 
to global systems). Moreover, these scales are hierarchically nested: micro-level 
interactions aggregate into meso-level patterns, which constitute macro-level structures.

In datafication, we must attend to the scale and hierarchical level at which we observe 
entities and their associated concepts, i.e., states, processes, properties, relations, and events. 
A phenomenon characterized at one scale may not be meaningfully represented at another.

% Individuals, groups
% Cross-sectional, longitudinal, panel
% Boundary specification

\subsection{Representations}

Having established what we dataficate (i.e., entities and their associated concepts) and 
their spatiotemporal characteristics, we now address how 
these are represented in quantitative research. We examine each component separately, 
attending to both their representational forms and scale-hierarchical considerations.

\subsubsection{Entities}

Entities are the fundamental units of observation in social research, the objects that 
possess states, undergo processes, exhibit properties, enter into relations, and 
participate in events. In Li's framework~\cite{li2019guoji}, entities constitute the 
ontological foundation upon which the five concepts operate.

In quantitative data, entities are typically represented through 
identification and categorical classification. Entities can be identified through unique 
identifiers or through combinations of properties that distinguish one entity from another. 
Beyond identification, entities are classified into types through categorical variables 
distinguishing individual versus collective actors, public versus private organizations, 
or state versus non-state actors. Some research designs represent entities through their 
attributes in multidimensional feature spaces, where each entity becomes a vector of 
characteristics.

Category theory provides a formal framework for reasoning 
about entity equality and similarity~\cite{awodey2010category}. Two entities are equal 
if they are isomorphic, meaning there exists a structure-preserving bidirectional mapping 
between them. More commonly in empirical research, we deal with similarity rather than 
equality: entities are similar if there exist morphisms, or structure-preserving mappings, 
between them. This framework is practically important for data collection: it guides 
decisions about which entities in the real world should be treated as instances of the 
same type and thus included in our dataset. For example, when studying organizations, we 
must determine whether non-profit entities, governmental agencies, and for-profit 
corporations are sufficiently similar morphologically to be compared, or whether their 
structural differences require separate treatment.

Entities exist at multiple scales and are often 
hierarchically nested. Individual persons nest within households, which nest within 
communities, which nest within nations. Organizations have departments, divisions, and 
subsidiaries. This nesting creates representational challenges: the same phenomenon may 
involve entities at different levels, such as individual voting behavior, party strategy, 
and national electoral systems simultaneously. Researchers must specify the primary unit 
of analysis while acknowledging cross-level interactions. Aggregating from lower to higher 
levels, such as from individual opinions to public opinion, involves assumptions about 
emergence and composition. Disaggregating from higher to lower levels risks ecological 
fallacy.



\subsubsection{States}

States refer to the conditions or situations that entities occupy at particular points in 
time. A state is a snapshot characterization: the configuration of an entity at a given 
moment. States can be simple or complex, capturing single attributes or multidimensional 
configurations.

In quantitative data, states are typically represented through 
variables measured at specific time points. For categorical states, nominal or ordinal 
variables capture discrete conditions such as employed versus unemployed, democratic 
versus authoritarian, or conflict versus peace. For continuous states, numerical variables 
measure magnitudes such as GDP, temperature, or public approval ratings. Complex states 
may be represented through vectors or matrices capturing multiple simultaneous attributes, 
such as a nation's economic, political, and social conditions at time t. In some frameworks, 
states are represented as probability distributions over possible configurations rather 
than point estimates, acknowledging uncertainty or heterogeneity.

States manifest at multiple scales and levels. 
Micro-states describe individual-level conditions, such as a person's employment status. 
Meso-states characterize group or organizational conditions, such as a firm's financial 
health. Macro-states describe system-level configurations, such as a nation's regime type. 
The relationship between states at different levels is not straightforward: macro-states 
are not simple aggregations of micro-states but may exhibit emergent properties. For 
example, a society can be in a state of high inequality even when most individuals are 
in similar economic states. Temporal granularity also matters: a state defined over 
microseconds differs fundamentally from one defined over decades. The choice of temporal 
resolution affects what patterns become visible and what variations are smoothed away.

States are inherently reductive, capturing entities in frozen moments. This 
temporal slicing obscures continuous flux and transitional phases. Many social phenomena 
resist clear state classification: is a society undergoing revolution in a stable state 
or between states? States also privilege measurable attributes while marginalizing 
qualitative characteristics that resist quantification. The decision of which attributes 
constitute the relevant state description embeds theoretical assumptions about what matters.

\subsubsection{Processes}

Processes refer to the temporal dynamics through which entities transition between states 
and develop properties. Unlike states which are snapshots, processes capture how entities 
evolve over time. A process is inherently temporal and directional, describing trajectories 
rather than positions. This conception is isomorphic to dynamical systems in physics, where 
processes correspond to trajectories through phase space (states) and parameter space 
(properties).

We distinguish processes from event emission and process generation, which we consider 
aspects of states rather than processes themselves. At any given time t, an entity's state 
may include its propensity to emit events or spawn sub-processes, but these emissions 
are state characteristics, not the process. The process is the evolution of these states 
and properties over time.

Representational forms. In quantitative data, processes are typically represented through 
time series, longitudinal measurements, or transition models. Time series capture repeated 
measurements of variables over successive time points, revealing trends, cycles, or 
fluctuations. Transition matrices represent discrete state changes, showing probabilities 
of moving from one state to another. Differential equations or difference equations model 
continuous processes mathematically, describing rates of change as trajectories through 
state space. Growth curves, decay functions, and trajectory models characterize specific 
process patterns. Phase space diagrams show how multiple state variables co-evolve, with 
processes appearing as curves or flows in this space.

Processes resist complete datafication because continuous temporal flow must 
be discretized into measurement intervals. The choice of sampling frequency determines 
what process characteristics are captured and what are aliased or missed entirely. Many 
social processes are non-stationary, meaning their dynamics change over time, violating 
assumptions of many analytical methods. Processes involving feedback loops, emergence, 
and non-linearity challenge simple representational schemes. Qualitative transformations, 
tipping points, and regime shifts may not be adequately captured by gradual quantitative 
change measures.

\subsubsection{Properties}

Properties are characteristics or attributes that entities possess. Unlike states which 
describe conditions at particular times, properties are relatively stable features that 
characterize entities across contexts. Properties can be intrinsic, belonging to the 
entity itself, or relational, defined through comparison with other entities.

In quantitative data, properties are represented through variables 
that characterize entities. Categorical properties use nominal variables to denote types, 
such as gender, nationality, or organizational form. Ordinal properties capture ranked 
characteristics, such as education level or firm size categories. Continuous properties 
use numerical variables to measure magnitudes, such as age, wealth, or geographic area. 
Composite properties combine multiple indicators into indices or scales, such as 
socioeconomic status or state capacity indices. Latent properties not directly observable 
are inferred through multiple manifest indicators using techniques like factor analysis 
or item response theory. In multidimensional representations, properties define coordinate 
systems in feature spaces where entities are positioned.
 
Properties can be defined at multiple levels and 
may not aggregate straightforwardly across scales. Individual-level properties such as 
education are distinct from group-level properties such as average education, which is 
distinct from distributional properties such as educational inequality. Some properties 
are emergent, existing only at higher levels of organization: a network has centralization 
properties that individual nodes do not possess. Other properties are contextual, defined 
relative to the surrounding environment: a person's relative income depends on the income 
distribution of their reference group. The ecological fallacy warns against inferring 
individual properties from aggregate properties, while the atomistic fallacy warns against 
inferring collective properties from individual attributes. Properties may also be 
scale-dependent: organizational complexity measured at the department level differs from 
complexity measured at the enterprise level.

Many theoretically important properties resist quantification. Concepts like 
legitimacy, identity, or cultural meaning are difficult to reduce to numerical indicators 
without significant loss. The operationalization of properties through specific indicators 
always involves construct validity concerns: does the measure actually capture the 
theoretical concept? Properties that are fluid, contested, or context-dependent challenge 
stable measurement. The reification of properties through measurement can obscure their 
socially constructed nature, treating as natural what is actually historical and contingent.

\subsubsection{Relations}

Relations describe connections, associations, or interactions between entities. Unlike 
properties which characterize individual entities, relations are inherently multi-entity 
concepts. Relations can be directed or undirected, symmetric or asymmetric, binary or 
multi-way.

In quantitative data, relations are represented through various 
relational structures. Dyadic relations between pairs of entities are captured in adjacency 
matrices, edge lists, or relational databases. Network representations use graphs where 
nodes represent entities and edges represent relations, with edge weights indicating 
relation strength. Multi-way relations involving more than two entities can be represented 
through hypergraphs or tensor structures. Relational attributes capture characteristics 
of the relations themselves, such as tie strength, duration, or type. Temporal networks 
track how relations change over time. Bipartite or multi-mode networks represent relations 
between different types of entities. Hierarchical or nested relations are represented 
through tree structures or multilevel network models.

Relations exist at multiple scales and form 
hierarchical structures. Micro-relations connect individuals through friendships, 
conversations, or transactions. Meso-relations link organizations through partnerships, 
supply chains, or alliances. Macro-relations connect nations through trade, treaties, or 
conflicts. These levels are interconnected: individual diplomatic interactions constitute 
interstate relations; organizational partnerships create industry structures. Relations 
themselves can be nested: individuals embedded in groups which are embedded in 
organizations which are embedded in broader institutional fields. The structure of 
relations at one level constrains and enables relations at other levels. Aggregating 
from micro-relations to macro-relations involves questions about how individual ties 
constitute systemic structures. Network properties like density, centralization, or 
clustering may differ fundamentally across scales.

Relational data faces unique challenges. Boundary specification is critical: 
defining which entities and which types of relations to include fundamentally shapes the 
resulting analysis. Many important relations are latent, informal, or difficult to observe 
directly. Relations may be multiplexed, with multiple types of ties between the same 
entities, complicating representation. Temporal dynamics of relation formation and 
dissolution are often inadequately captured in cross-sectional network data. The meaning 
and significance of relations can be context-dependent and not fully captured by 
structural position alone. Power relations, symbolic relations, and relations of meaning 
often resist reduction to measurable ties.

\subsubsection{Events}

Events are discrete occurrences that happen at specific points or intervals in time. 
Unlike processes which describe continuous change, events are bounded happenings: 
transitions, occurrences, or incidents that mark temporal discontinuities.

In quantitative data, events are represented through temporal 
markers and occurrence indicators. Binary variables indicate whether an event occurred 
during a given period. Timestamps record exact timing of events. Event counts aggregate 
how many times an event type occurred. Duration variables measure how long events lasted. 
Event sequences track ordered series of occurrences. Survival or hazard models represent 
time-to-event data, analyzing when events occur and what factors affect their timing. 
Point process models treat events as points in time with associated intensities. Event 
attributes capture characteristics of specific occurrences, such as magnitude, location, 
or participants. Complex events may be decomposed into event structures showing 
sub-events and their relationships.

Events occur at multiple temporal and organizational 
scales. Micro-events are brief and localized, such as a single speech act or transaction. 
Macro-events span extended periods and broad scope, such as wars, revolutions, or economic 
crises. Events at different scales may be hierarchically related: micro-events can 
constitute or trigger macro-events, while macro-events provide contexts that shape 
micro-events. A revolution consists of countless individual acts of resistance, protests, 
and confrontations, yet the revolution as macro-event has properties and consequences not 
reducible to its component micro-events. The temporal granularity of event measurement 
affects what is visible: daily event data capture fluctuations that monthly aggregates 
smooth away. Events can also cascade across scales: a local bank failure may trigger 
regional financial instability which precipitates a national crisis.

Event identification and boundaries are often ambiguous. When exactly does 
an event begin and end? What counts as a distinct event versus a continuation or 
recurrence? Many significant events leave minimal empirical traces amenable to 
datafication. The timing resolution of event data affects analysis: recording only the 
date versus the exact second of occurrence provides different analytical possibilities. 
Rare or unprecedented events challenge statistical approaches developed for recurring 
patterns. The interpretation of events depends on theoretical framing: the same occurrence 
may be categorized as different event types depending on analytical perspective. Events 
may be socially constructed, with their recognition and classification reflecting power 
and interpretive struggles rather than objective happenings.

